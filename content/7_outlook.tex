\chapter{Fazit}
\label{ch:summary}

Zusammenfassend lässt sich sagen, dass die Entwicklung des Prototyps erfolgreich war und nach einem Review durch andere Entwickler auch auf Crossload live geschaltet werden kann.
Alle gestellten Anforderungen wurden umgesetzt und sind erfolgreich lokal getestet worden.
Durch Betrachtung der theoretischen Grundlagen der Relevanz und der vorhandenen Dokumentation von z. B. Solr konnten bereits vorhandene Erkenntnisse in die Entwicklung einfließen.

Durch diesen Prototyp konnte bereits existierende Funktionalität, die Suche nach Inhalten erweitert werden.
Außerdem wird dem Nutzer ein Mehrwert in Form von relevanteren Inhalten auf der Suchergebnisseite geboten.
Durch eine Wiederverwendung von Komponenten im Frontend wurde die Übersichtlichkeit und Wartbarkeit der Anwendung nicht verletzt.

Die größten Schwierigkeiten bzw. Aufwände dieser Entwicklung lagen in der theoretischen Ausarbeitung der Grundlagen der Relevanz, um geeignete Methoden zu finden, mit welchen die existierende Relevanz verbessert werden konnte.
Außerdem mussten geeignete Punkte in der Implementation der Suche und im Frontend gefunden werden, um Änderungen vorzunehmen, ohne bereits funktionsfähige Programmteile einzuschränken.

Der entwickelte Prototyp könnte noch durch automatisierte Testfälle verbessert werde, welche aber derzeit in der Such API und im Webfrontend nur minimal vorliegen.
Der Grund hierfür ist die begrenzte Zeit, die die ehrenamtlichen Entwickler in das Projekt einbringen können.
Statt automatisierte Testfälle auszuarbeiten, werden derzeit neue Funktionen höher priorisiert.

Dieser Prototyp ist hierbei in der Suche von Crossload nur ein kleiner Teil einer ganzen Kette von Verbesserungen und Änderungen, die vorgenommen werden, um den Nutzern relevantere Inhalte zu bieten.
