\chapter{Auswertung}
\label{ch:evaluation}
Ziel des Prototyps war es, für die Website Crossload relevantere Inhalte in der Suche herauszufiltern und anzuzeigen.
Im Verlaufe dieses Kapitels werden die konzeptionierten und entwickelten Ergebnisse genauer untersucht.

Anhand des Prototyps und den erhobenen funktionalen Anforderungen kann einfach festgestellt werden, inwiefern diese Funktionalitäten für den Nutzer möglich sind.
Für den Abgleich dieser Funktionalitäten wird für jede Anforderung ein kurzer Titel gegeben, der Status, ob diese erfüllt wurde oder nicht, sowie falls eine Begründung oder Zwischenstand der Bearbeitung.

\begin{longtable}{p{0.25\textwidth}|c|p{0.5\textwidth}}
  \label{tab:requirements}\\
  \textbf{Anforderung} & \textbf{Status} & \textbf{Begründung} \\
  \hline
  \hline
  Gemischte Suchergebnisse über alle Kategorien & Erledigt & Suchergebnisse werden zusammengeführt auf der Suchergebnisseite angezeigt. \\
  \hline

  Relevanz für ähnliche Schlagwörter & Gegeben & Ähnliche Suchtitel werden bereits durch *-Zeichen in der gegebenen Suche gefunden (Vgl. \ref{code:SOLRSuggestionQuery}). \\
  \hline

  Relevanz anhand von Kategorien & Erledigt & Für eingegebene Kategorien werden ähnliche Inhalte um ein vielfaches geboostet. \\
  \hline

  Nach Relevanz absteigende Sortierung & Gegeben & Sortierung und Richtung bereits auf Suchergebnisseite gegeben \\
  \hline

  Vorschläge & Erledigt & Der relevanteste Vorschlag wird von der Suche herausgefiltert und falls gefunden, auf der Suchergebnisseite angezeigt. \\
  \caption{Anforderungsanalyse}
\end{longtable}

Der endgültige Stand aller funktionalen Anforderung kann in folgenden Kennzahlen zusammengefasst werden:
\begin{itemize}
  \item 3 von 5 Anforderungen wurden erfüllt.
  \item 2 von 5 Anforderungen waren bereits gegeben und wurden nicht verändert.
\end{itemize}
