\chapter{Konzeption}
\label{ch:conception}
Bevor mit der Entwicklung des Prototyps gestartet werden kann, geht die Planung und Konzeption der Erweiterung voraus.
Der Entwurf einer Software ist die Basis für jede Entwicklung.
Anfangs wird die momentane Anwendung auf bereits implementierte Funktionalität überprüft und schließlich mithilfe der erarbeiteten Methoden zur Bewertung der Relevanz auf Grundlage der gesammelten Anforderungen verbessert.

\section{Bisherige implementierte Funktionalität}
\label{sec:implementedFunctionality}
Bei \gls{crossload} werden verschiedene Typen bzw. Kategorien von Inhalten in der von SOLR indizierten Datenbank über eine Spring Boot Anwendung an das Webfrontend zur Verfügung gestellt.
Der initiale und derzeit implementierte Gedanke dabei ist, die Inhalte auch in diesen Kategorien zu übertragen und in fester Reihenfolge anzuzeigen.
Diese Vorgehensweise hat jedoch einige Nachteile:

\begin{itemize}
  \item \textbf{Relevanz:} Der möglicherweise relevanteste Inhalt wird nicht als erstes angezeigt, da dessen Kategorie relativ weit unten angezeigt wird.
  \item \textbf{Übersicht:} Es ist schwer für den Nutzer eine Übersicht über alle gefundenen Inhalte zu erlangen.
  \item \textbf{User Experience:} Höchstwahrscheinliche Treffer (90-100 \% Trefferwahrscheinlichkeit) werden nicht direkt vorgeschlagen.
\end{itemize}

Diese Nachteile sollen im Verlaufe der Entwicklung verbessert werden.
Ebenso sollen auch die bisher genutzten Methoden zur Berechnung der Relevanz verbessert werden.
Diese umfassen derzeit:

\begin{itemize}
  \item Text Matching auf verschiedene Textteile und Attribute. Hier werden verschiedene Attribute in 3 Kategorien (hoch, mittel, niedrig) wie folgend bewertet:
  \begin{itemize}
    \item \textbf{Hoch:} Titel, Serie, Thema, Autor
    \item \textbf{Mittel:} Untertitel, Schlagwörter, Kategorie, Thema
    \item \textbf{Niedrig:} Verlag, Standort, Dateiname, Speech to Text, Mitschrift, Suchsnippet
  \end{itemize}
  \item Matching des Suchterms zu einem Bibelvers.
  \item Oder falls kein Suchterm gegeben ist, werden Inhalte mit Video oder Bild höher bewertet.
  \item Filter für mitgegebene Query Parameter: Kategorie, Serie, Event, Thema, Jahreszahl oder Dauer. Inhalte, die nicht zu diesem Filter passen, werden komplett aussortiert.
\end{itemize}

\section{Verbesserungen für relevantere Inhalte}
\label{sec:potential}

Grundsätzlich findet die Anwendung bereits passende bzw. relevante Inhalte durch das Matching der verschiedenen Attribute (\ref{sub:relevanceAttribute}) und Textteile (\ref{sub:relevanceText}).
Ebenso das Matching bezüglich des Bibelverses oder des initialen Boosting über ein vorhandenes Video oder Bild führt bereits zum gewünschten Ergebnis und eine Änderung würde hier keinen nennenswerten Mehrwert bieten.

Dennoch gibt es einige Ideen, relevantere Inhalte für den Nutzer herauszugeben: zusammengeführte Listen, ein Schlagwortabgleich auf die Kategorien und Vorschläge zur weiteren Navigation.

\subsection{Zusammengeführte Liste}
\label{sub:unifiedList}
Als oberstes Ziel wird die Liste der gefundenen Inhalte, momentan aufgespalten in die verschiedenen Kategorien wie z. B. Bild, Video, Predigt, Buch, etc., in eine große Liste überführt.
Dadurch können relevante Inhalte, die bisher durch die vordefinierte Sortierung der Kategorien auf der Suchseite nicht als erste aufgelistet wurden, an der Stelle angezeigt werden, an die der Nutzer sie erwartet.

Damit der Nutzer dennoch sieht, welcher Inhalt welche Kategorie, wird anschließend zu der ausführlichen Version des Ergebnisses ein kleiner Text mit dessen Kategorie hinzugefügt.
So geht die bisherige Funktionalität nicht komplett verloren und der Nutzer erhält die relevantesten Inhalte direkt an erster Stelle und sieht sofort dessen Kategorie.

\subsection{Schlagwortabgleich}
\label{sub:keyword}
Eine Möglichkeit, die Relevanz der Suchergebnisse im Sinne der Aufgabenstellung zu verbessern, wäre es, anhand des Suchbegriffes herausfiltern, ob z. B. ein Schlagwort wie „Video“ oder „Bild“ verwendet wurde und somit relevante Inhalte dieser Kategorie höher zu gewichten.

Dafür müssten relevante Schlagwörter ermittelt werden und auch in allen möglichen Varianten untersucht werden, um ein hilfreiches Matching zu erhalten, welches dann anhand dem Attribute „Hauptkategorie“ nachvollzogen werden kann.
Eine gewisse Menge an Varianten kann vordefiniert werden, um einen Großteil der Anfragen korrekt abzufangen.
Um letztendlich aber eine mehr und mehr vollständige Menge an Varianten und Suchbegriffen zu erhalten müssen die abgegebenen Suchabfragen untersucht werden.
Diese können aber mit Matomo untersucht werden und mit der Zeit angepasst werden.\footnote{Siehe \ref{sec:crossload} \cite{matomo2022}}

Für den Start wären folgende Schlagwörter für die vorhandenen Kategorien denkbar:\footnote{Siehe Duden \cite{dudensynonyme2022}}
\begin{itemize}
  \item \textbf{Predigten (mit Video):} Video, Film, Stream, Live, Livestream.
  \item \textbf{Predigten (mit und ohne Video):} Predigt, Vortrag, Mahnwort.
  \item \textbf{Bücher:} Buch, Bücher, Taschenbuch, Sammelband, Reader, Druck, Bestseller.
  \item \textbf{Bilder:} Bild, Darstellung, Zeichnung, Aufnahme, Foto, Fotografie.
  \item \textbf{Musik:} Song, Melodie, Hymne, Stück, Gesang, Klavier, Musik, Orchester.
  \item \textbf{Hörbücher:} Hörbuch, Hörbücher, Audiobook.
  \item \textbf{Sonstige:} Sonstige.
\end{itemize}

\subsection{Vorschläge für weitere Navigation}
\label{sub:suggestion}
Optimalerweise gibt der Nutzer eine Suchanfrage ein, zu der ein Inhalt eine sehr hohe Relevanz hat und alle anderen Inhalte eine recht niedrige.
Sollte dies der Fall sein, so könnte dieser Inhalt in einer Vorschlagsbox über der Ergebnisliste angezeigt werden, damit der Nutzer visuell sieht, dass dies der Inhalt ist, den er höchstwahrscheinlich sucht.
Eine Berechnung hierfür ist nicht klar definiert, als ersten Versuch wird überprüft, ob der berechnete Score des relevantesten Inhalts mindestens doppelt so groß ist, wie der des nächsten Inhalts.
Dieses Vorgehen muss aber in der Entwicklung und im produktiven Betrieb weiter geprüft werden, um dieses Vorgehen weiter zu verfeinern.

Eine mögliche Schwachstelle hierbei könnten sehr relevante Inhalte direkt am Anfang sein, bei denen die darauf folgenden Inhalte im Vergleich irrelevant sind.
Somit könnten auch beide Inhalte vorgeschlagen werden, was aber aus Gründen der Nutzerfreundlichkeit nur auf einen minimiert wird.
Dafür müsste die ganze Liste, oder ein Teil z. B. die Top 10, auf die durchschnittliche Relevanz geprüft werden und Inhalte, die sich stark nach oben von diesem Durchschnitt unterscheiden, als Vorschläge genommen werden.

Für die Implementierung wird hierbei zuerst geprüft, ob ein Inhalt vorgeschlagen werden kann. Falls dies nicht zutrifft, wird anschließend mit der Prüfung des Durchschnitts fortgefahren.
