\chapter{Konzeption}
\label{ch:conception}
Bevor mit der Entwicklung des Prototyps gestartet werden kann, geht die Planung und Konzeption der Erweiterung voraus.
Der Entwurf einer Software ist die Basis für jede Entwicklung.
Anfangs wird die momentane Anwendung auf bereits implementierte Funktionalität überprüft und schließlich mithilfe der erarbeiteten Methoden zur Bewertung der Relevanz auf Grundlage der gesammelten Anforderungen verbessert.

\section{Bisherige implementierte Funktionalität}
\label{sec:implementedFunctionality}
Bei \gls{crossload} werden verschiedene Typen bzw. Kategorien von Inhalten in der von SOLR indizierten Datenbank über eine Spring Boot Anwendung an das Webfrontend zur Verfügung gestellt.
Der initiale und derzeit implementierte Gedanke dabei ist, die Inhalte auch in diesen Kategorien zu übertragen und in fester Reihenfolge anzuzeigen.
Diese Vorgehensweise hat jedoch einige Nachteile:

\begin{itemize}
  \item \textbf{Relevanz:} Der möglicherweise relevanteste Inhalt wird nicht als erstes angezeigt, da dessen Kategorie relativ weit unten angezeigt wird.
  \item \textbf{Übersicht:} Es ist schwer für den Nutzer eine Übersicht über alle gefundenen Inhalte zu erlangen.
  \item \textbf{User Experience:} Höchstwahrscheinliche Treffer (90-100 \% Trefferwahrscheinlichkeit) werden nicht direkt vorgeschlagen.
\end{itemize}

Diese Nachteile sollen im Verlaufe der Entwicklung verbessert werden.
Ebenso sollen auch die bisher genutzten Methoden zur Berechnung der Relevanz verbessert werden.
Diese umfassen derzeit:

\begin{itemize}
  \item Ein von SOLR schon verfügbares Matching auf Textteile.
  \item Ein Matching des Suchterms zu einem Bibelvers.
  \item Suche nach dem Autor eines Inhalts.
  \item Oder falls kein Suchterm gegeben ist, werden Inhalte mit Video oder Bild höher bewertet.
  \item Filter für mitgegebene Query Parameter: Kategorie, Serie, Event, Thema, Jahreszahl oder Dauer.
\end{itemize}

\section{Verbesserungspotenzial für relevantere Inhalte}
\label{sec:potential}


\section{Planung der Implementation}
\label{sec:planning}
