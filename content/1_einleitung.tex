\chapter{Einleitung}\label{ch:intro}

\section{Relevanz des Themas}
Suchalgorithmen und relevante Suchergebnisse sind derzeit so relevant wie noch nie. Dabei wollen die Benutzer einer Suchmaschine in Sekundenbruchteilen Ergebnisse, die am besten zu ihrem Suchbegriff passen, ohne sich dabei viel Gedanken über die Formulierung eines solchen Begriffes zu machen. Ein Beispiel für einen solchen Algorithmus ist Google, welches seit den frühen 2000ern einen kometenhaften Aufstieg in der Welt der Suchmaschinen hinter sich hat, was anhand der erreichten Werbeeinnahmen sichtbar wird.\cite{googleUmsatz} Google ist im Vergleich zu anderen Suchmaschinen so stark verbreitet,\cite{googleShare} dass mittlerweile sogar der Duden das Verb „googeln“ als eigenen Begriff für die Recherche im Internet führt.\cite{dudenGooglen}
Dabei stellt sich für die Entwicklung eigener Produkte die Frage, wie aus einem Suchbegriff, der meist nur aus wenigen Wörtern bis zu einem ganzen Satz besteht, relevante Suchergebnisse gefunden werden können. Dies würde zur Akzeptanz der Nutzer im Hinblick auf die entwickelte Funktionalität führen, da gewünschte Ergebnisse schneller und ohne großen Aufwand gefunden werden können.

\section{Ausgangssituation}
Derzeit besteht bei Crossload4, einer Plattform zum Durchsuchen und Anhören einer umfassenden Predigtdatenbank, eine Datenbank mit einer Such \gls{api} auf Basis von Spring Boot und \gls{solr}. Diese teilt auf der Suchergebnisseite die Ergebnisse nach Kategorien auf und somit können nur schwer übergreifende Suchanfragen getätigt werden. Zwar werden alle Treffer auf der gleichen Seite angezeigt, doch durch die Aufteilung nach Kategorien werden Ergebnisse gewisser Kategorien über anderen gezeigt, auch wenn niedrig positionierte Kategorien relevantere Ergebnisse enthalten.

\section{Zielsetzung}
Das Ziel der vorliegenden Studienarbeit ist es durch eine theoretische Betrachtung der Bewertung der Relevanz von Suchergebnissen und der anschließenden Entwicklung eines Prototypen, ein bestehendes Produkt zu erweitern. Diese Erweiterung umfasst, die nach Kontext und Wahrscheinlichkeiten gewichtete und gefilterte Suche über eine Datenbank mit Datentypen verschiedener Kategorien bei der zusätzlich Vorschläge zur weiteren Navigation auf der Suchergebnisseite gegeben werden sollen.
