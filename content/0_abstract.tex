\thispagestyle{empty}
\section*{Kurzdarstellung}
\label{sec:kurzdarstellung}
Das Ziel der vorliegenden Studienarbeit ist es, eine bestehende Datenbank mit multimedialen Inhalten möglichst effizient nach unterschiedlichen Kriterien zu durchsuchen.
Suchergebnisse sollen nach bestimmten Kriterien gewichtet, gefiltert und sortiert werden. Vorschläge für eine weiterführende Navigation auf der Suchergebnisseite sollen angeboten werden, Suchergebnisse sollen dazu nach Kontext und Wahrscheinlichkeiten gewichtet angezeigt werden.
Das theoretische Fundament dieser Arbeit stellt die wissenschaftliche Betrachtung der Methoden zur Bewertung der Relevanz von Suchergebnissen dar. Die Arbeit untersucht die Möglichkeit, einen Suchbegriff so zu analysieren, dass ein Nutzer die bestmögliche Ergebnisliste bzw. zielgerichtete weiterführende Navigationsmöglichkeiten erhält.
Die bestehende Anwendung „Crossload“ wird vorgestellt, um dem Leser einen Kontext zu bieten, in der sich die Entwicklung bewegt.

\section*{Abstract}
\label{sec:abstract}
The goal of this student research project is to search an existing database with multimedia content as efficiently as possible according to various criteria.
Search results are to be weighted, filtered, and sorted according to certain criteria. Suggestions for further navigation on the search results page are to be offered, and search results are to be displayed weighted according to context and probabilities.
The theoretical foundation of this work is the scientific consideration of methods for evaluating the relevance of search results. The work examines the possibility of analyzing a search term in such a way that a user receives the best possible list of results or targeted further navigation options.
The existing application "Crossload" is presented to provide the reader with a context in which the development takes place.

\cleardoublepage

\section*{Eidesstattliche Erklärung}
\label{sec:explanation}
Hiermit versichere ich, Marc Jonas Roser, ehrenwörtlich, dass ich die vorliegende Studienarbeit mit dem Titel: „Entwicklung eines Prototypen eines Suchalgorithmus zur Bewertung von Suchergebnissen verschiedener Kategorien“ selbstständig und ohne fremde Hilfe verfasst und keine anderen als die angegebenen Hilfsmittel benutzt habe. Die Stellen der Arbeit, die dem Wortlaut oder dem Sinn nach anderen Werken entnommen wurden, sind in jedem Fall unter Angabe der Quelle kenntlich gemacht. Die Arbeit ist noch nicht veröffentlicht oder in anderer Form als Prüfungsleistung vorgelegt worden.
Ich versichere zudem, dass die eingereichte elektronische Fassung mit der gedruckten Fassung übereinstimmt.

Nürnberg, 25.09.2022

Marc Jonas Roser
